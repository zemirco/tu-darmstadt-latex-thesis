Table \ref{tab:inflation} shows a simple example with footnotes.

\ctable[
	cap = Economic situation in Austria,
	caption = Economic situation in Austria \cite{verbraucher2011, prognose2011},
	label	= tab:inflation,
	pos = htb
	]{lrr}
	{
		\tnote[1]{Consumer price index}
		\tnote[2]{Index 1990 = 100}
		\tnote[3]{Forecast value}
	}
	{\FL	Year			& Inflation rate in \%\tmark[1]		& Price trend\tmark[2]
	\ML		2000			& 2,3								& 125,7
	\NN		2001			& 2,7								& 129,1
	\NN		2002			& 1,8								& 131,5
	\NN		2003			& 1,3								& 133,2
	\NN		2004			& 2,1								& 136,0
	\NN		2005			& 2,3								& 139,1
	\NN		2006			& 1,5								& 141,2
	\NN		2007			& 2,2								& 144,3
	\NN		2008			& 3,2								& 148,9
	\NN		2009			& 0,5								& 149,6
	\NN		2010			& 1,9								& 152,4
	\NN 	2011\tmark[3]	& 2,1								& 155,6
	\NN		2012\tmark[3]	& 1,8								& 158,4
	\LL}

\newpage

Table \ref{tab:data_imiev} shows a more complex example

\ctable[
	caption = Technical data of the Mitsubishi i-MiEV \cite{imiev_daten},
	cap = Technical Data of the Mitsubishi i-MiEV,
	label	= tab:data_imiev,
	pos = htb
	]{llr}
	{\tnote[1]{Data measured in NEDC}}
	{\FL Description & Specification & Value
		\ML \multirow{4}*{Dimensions} & Length & 3475 mm
		\NN & Width & 1475 mm
		\NN & Height & 1610 mm
		\NN & Wheel base & 2550 mm
		\ML \multirow{5}*{Load capacity} & Luggage space & 227 / 860 liters
		\NN & Curb weight & 1110 kg
		\NN & Permissible maximum weight & 1450 kg
		\NN & Payload & 340 kg
		\NN & Seats & 4 People
		\ML \multirow{3}*{Driving characteristics} & Energy consumption & 135 Wh/km
		\NN & Range & 150 km\tmark[1]
		\NN & max. Velocity & 130 km/h
		\ML \multirow{5}*{Battery data} & Nominal voltage & 330 V
		\NN & Electric charge & 50 Ah
		\NN & Theoret. energy & 16500 Wh
		\NN & Mass & 165 kg
		\NN & Energy density & 100 Wh/kg
		\ML \multirow{3}*{Motor data} & Typ & Permanent Synchronous Motor
		\NN & Nominal power & 35 kW
		\NN & max. Torque & 180 Nm
		\LL}